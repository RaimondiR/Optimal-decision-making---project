\documentclass[a4paper]{article}

\usepackage[english]{babel}
\usepackage[utf8]{inputenc}
\usepackage{amsmath,amssymb}
\usepackage{gensymb}
\usepackage{graphicx}
\usepackage{comment}
%\usepackage[colorinlistoftodos]{todonotes}
\usepackage{float}
\usepackage[left=2cm,right=2cm,bottom=2cm,includefoot]{geometry} % pour les marges
%\usepackage{sidecap} % pour avoir des titres de fig/table sur les côtés
%\usepackage{anyfontsize}
\usepackage{hyperref}
\hypersetup{
    colorlinks   = true, % Colours links instead of ugly boxes
    urlcolor     = blue, % Colour for external hyperlinks
    linkcolor    = blue,% Colour of internal links
    citecolor    = blue   % Colour of citations
}
\usepackage[x11names]{xcolor}
%\usepackage{subfigure}
%\usepackage{listingsutf8}
\usepackage{listings} % pour insérer les codes
%\usepackage{textcomp}
%\usepackage{enumerate}
%\usepackage{multicols}
\usepackage{xspace}
\usepackage{caption}
%\usepackage{titlesec} % pour pouvoir modifier les titres des sections
\usepackage[bottom]{footmisc} % pour que les notes en bas de page colle le bas même si page vide
\usepackage[normalem]{ulem} % pour pouvoir barrer un texte quelconque
%\usepackage{clrscode3e} % pour les pseudo-codes
%\usepackage{tikz} % pour les arbres
\usepackage{tablefootnote} % Footnote in table -> use \tablefootnote{}
%\usepackage[nomessages]{fp} % see https://tex.stackexchange.com/questions/154224/fp-calculation-variable
\usepackage{diagbox}

%\renewcommand{\thesubsection}{\thesection.\alph{subsection}}
\newcommand{\eme}{$^{\text{ème}}$\xspace}
\newcommand{\rem}[1]{\textcolor{red}{[#1]}}
\newcommand{\para}[1]{\paragraph{#1}\quad\\}
\newcommand{\vsp}{\vspace{1em}}
\newcommand{\ds}{\displaystyle}
\renewcommand{\tt}[1]{\texttt{#1}\xspace}
\renewcommand{\u}[1]{\underline{#1}\xspace}
\newcommand{\ie}{\textit{i.e.}\xspace}
\renewcommand{\b}{$\bullet$\xspace}
\newcommand{\itemb}{\item[\b]}
\colorlet{mygreen}{green!60!black}
\newcommand{\newRem}[2]{\expandafter\newcommand\csname#1\endcsname[1]{\textcolor{#2}{[#1: ##1]}}}
\newRem{nico}{mygreen}
\newRem{remi}{blue}

%\newcommand{\ordi}[1]{\hyperref[ax:#1]{\includegraphics[scale=0.004]{ordi}}\\}
\newcommand{\code}[1]{\begin{minipage}{\textwidth}\lstinputlisting[label=ax:#1,title=$\mathtt{#1.m}$]{MatLab_codes/#1.m}\end{minipage}}
\lstset{
    inputencoding=utf8,  
    extendedchars=true,
    language=C,
    frame=single,
    breaklines=true,
    numbers=left,
    numbersep=5pt,
    basicstyle=\footnotesize\ttfamily,
    commentstyle=\itshape\color{gray},
    keywordstyle=\bf\color{mygreen},
    %identifierstyle=\color{black},
    %stringstyle=\color{black},
    %numberstyle=\color{red},
    %emph={read2Darrayshm,write2Darrayshm,send_message,read_message,writeshm,readshm,wait,waitAll,signal,signalAll},
    %emphstyle=\color{orange!70!red},
    morekeywords={foreach, interrupt},
    morecomment=[l][\itshape]{!!},
    showstringspaces=false,
    float,
    floatplacement=H,
    literate=%
        {à}{{\`a}}1
        {é}{{\'e}}1
        {è}{{\`e}}1
        {É}{{\'E}}1
        {î}{{\^i}}1
        {ô}{{\^o}}1
        {‰}{{\textperthousand}}1
        {µ}{{$\mu$}}1
}

\setlength{\parindent}{0cm}
